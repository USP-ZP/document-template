%%%%%%%%%%%%%%%%%%%%%%%%%%%%%%%%%%%%%%%%%%%%%%%%%%%%%%%%%%%%%%%%%%%%%%%%%%%%%%%%%%%%%%%%%%%%%%%
%  USP-DOC
%  Template para documentos técnicos das USP
%
%  Copyright (c) 2023 Estevão Soares dos Santos
%%%%%%%%%%%%%%%%%%%%%%%%%%%%%%%%%%%%%%%%%%%%%%%%%%%%%%%%%%%%%%%%%%%%%%%%%%%%%%%%%%%%%%%%%%%%%%%

% NOTA: Este ficheiro é o template e NÃO DEVE SER ALTERADO.
% Para alterar o conteúdo, editar os ficheiros da pasta "conteudos"
% AS informações acerca da recomendação estão no ficheiro "metadados.tex"

\documentclass[12pt]{article}

%! suppress = MultipleIncludes
\usepackage{luacode}
\usepackage{tabularx}
\usepackage[portuguese]{babel} % Language setting
\usepackage{csquotes}
\usepackage{titlesec} % Load the titlesec package for formatting section and subsection titles
\usepackage[a4paper,top=2cm,bottom=2cm,left=2cm,right=2cm,marginparwidth=1.75cm]{geometry} % Set page size and margins
\usepackage[T1]{fontenc} % fonts
\usepackage{Oswald} % Oswald
\usepackage{helvet} % Helvetica
\renewcommand*\familydefault{\sfdefault} % Makes Helvetica the default sans-serif font
\usepackage[backend=biber]{biblatex} % Bibliografia
\addbibresource{conteudo/referencias.bib}
\usepackage{amsmath}
\usepackage{graphicx}
\usepackage[hidelinks]{hyperref}
\usepackage[table]{xcolor}
\usepackage{tikz}
\usetikzlibrary{calc}
\usepackage[export]{adjustbox}
\usepackage{afterpage}
\usepackage{changepage}
\usepackage{lipsum}
\usepackage[shiftHeadings=1, startNumber=false, pipeTables=true, fencedCode]{markdown}
\usepackage{fancyhdr}
\usepackage{catchfile}
\usepackage{etoolbox}
\usepackage{subfiles} % Best loaded last in the preamble

% my packages
\usepackage{template/cabecalho}
%\usepackage{showframe}

\begin{luacode}
    dofile("./template/utils/parse_config.lua")
\end{luacode}

\directlua{parse_config("./conteudo/config.yml")}

%%%%%%%%%%%%%%%%%%%%%%%%%%%%%%%%%%%%%%%%%%%%%%%%%%%%%%%%%%%%%%%%%%%%%%%%%%%%%%%%%%%%%%%%%%%%%%%%
% INFORMAÇÔES:
%
% O número do procedimento deve ter o seguinte formato ID.ANOvMAJOR.MINOR.PATCH
%  - ID: número composto por 3 dígitos (ex: 001). O ID é incremental e faz reset em cada ano
%  - ANO: Ano civil em que o procedimento foi criado pela primeira vez
%  - MAJOR: Versão major do documento. Corresponde a uma grande revisão do documento.
%           Geralmente implica nova revisão cientifica extensa
%  - MINOR: Versão minor do documento. Corresponde a uma pequena revisão do documento, 
%           com alterações "cirurgicas" da recomendação
%  - PATCH: Versão patch do documento. Corresponde a pequenas alterações como correcção de 
%           gralhas ou erros ortográficos
%
%  NOTA: Nem todos os tipos de documentos usam todos os tipos de metadados


% escolher o tipo de documento apropriado, removendo o símbolo de comentário (%) do inicio da linha
% e comentando as restantes
\def\tipodocumento{documento}           % documento genérico
%\def\tipodocumento{recomendacao}       % recomendação
%\def\tipodocumento{procedimento}       % procdimento
%\def\tipodocumento{instrucaotrabalho}  % instrução de trabalho
%\def\tipodocumento{checklist}  % instrução de trabalho


% Variáveis do documento (alterar conforme necessário)
\def\titulo{Template para documento da USP com um título que pode ser bastante longo}
\def\id{001}
\def\ano{2023}
\def\major{1}
\def\minor{0}
\def\patch{0}

% palavras chave
\def\pchave{Rastreio, Enterobacterales produtores de carbapenemases, \emph{Staphylococcus aureus} resistente a 
	meticilina, admissão hospitalar, internamento} 

% destinatarios
\def\destinatarios{Profissionais de Saúde do ACES Oeste Norte}

\def\local{USP do ACES Oeste Norte}

\def\dataElaboracao{2023-09-01}
\def\dataRevisao{2023-12-31}
\def\dataAprovacao{2023-12-31}

\newcommand{\proposta}{
O presente documento foi elaborado por proposta da conjunta da 
Unidade de Saúde Pública do ACES Oeste Norte,
o Centro Hospitalar do Oeste e do Departamento de Saúde Pública
da Administração Regional de Saúde de Lisboa e Vale do Tejo.
}

\def\equipaRedatora{
	João Silva, Maria Fernandes, Bruno Almeida, Ana Luísa Pereira
}
\def\equipaRevisora{
	Tiago Carvalho, Sofia Martins, André Lopes Rocha
}
\def\equipaConsultora{
	Rita Oliveira Santos e Pedro Gonçalves
}


%\newcommand{\numero}{D\id.\ano}
%\newcommand{\versao}{\major.\minor.\patch}


%Esquema de Cores
\definecolor{gray1}{RGB}{240, 240, 240}
\definecolor{gray2}{RGB}{245, 245, 245}
\definecolor{red1}{RGB}{204, 0, 51} % recomendacao
\definecolor{blue1}{RGB}{0,103,153} % procedimento
\definecolor{blue2}{RGB}{0,145,145} % instrucao trabalho
\definecolor{green1}{RGB}{0,150,97}
\definecolor{orange1}{RGB}{224,73,0}


\let\oldaddcontentsline\addcontentsline
\newcommand{\stoptocentries}{\renewcommand{\addcontentsline}[3]{}}
\newcommand{\starttocentries}{\let\addcontentsline\oldaddcontentsline}

\newcommand{\nonumbersections}[1]{
    % Remove numbers from sections and subsections
    % format sections and subsections titles to be red
    \titleformat{\section}
    {\normalfont\sffamily\Large\bfseries\color{#1}}
    {}{0em}{}
    
    \titleformat{\subsection}
    {\normalfont\sffamily\normalsize\bfseries\color{#1}}
    {}{0em}{}
    
    \titleformat{\subsubsection}
    {\normalfont\sffamily\normalsize\bfseries\color{#1}}
    {}{0em}{}
}

\newcommand{\numberedsections}[1]{
    % Remove numbers from sections and subsections
    % format sections and subsections titles to be red
    \titleformat{\section}
    {\normalfont\sffamily\Large\bfseries\color{#1}}
    {\thesection.}{1em}{}
    
    \titleformat{\subsection}
    {\normalfont\sffamily\normalsize\bfseries\color{#1}}
    {\thesubsection.}{1em}{}
    
    \titleformat{\subsubsection}
    {\normalfont\sffamily\normalsize\bfseries\color{#1}}
    {\thesubsubsection.}{1em}{}
}


% definition to store the type of document
\newcommand{\docname}{Documento}

%! suppress = NonMatchingIf
\ifdefstring{\tipodocumento}{recomendacao}{
    \colorlet{colorschema}{red1}
    \renewcommand{\docname}{Recomendação}
    \renewcommand{\numero}{R\id.\ano}
    \nonumbersections{colorschema}
}{
    \ifdefstring{\tipodocumento}{procedimento}{
        \colorlet{colorschema}{blue1}
        \renewcommand{\docname}{Procedimento}
        \renewcommand{\numero}{P\id.\ano}
        \numberedsections{colorschema}
    }{
        \ifdefstring{\tipodocumento}{instrucaotrabalho}{
            \colorlet{colorschema}{blue2}
            \renewcommand{\docname}{Instrução de Trabalho}
            \renewcommand{\numero}{T\id.\ano}
            \numberedsections{colorschema}
        }{
            \ifdefstring{\tipodocumento}{checklist}{
                \colorlet{colorschema}{orange1}
                \renewcommand{\docname}{Lista de Verificação}
                \renewcommand{\numero}{C\id.\ano}
                \nonumbersections{colorschema}
            }{
                \colorlet{colorschema}{green1}
                \renewcommand{\docname}{Documento}
                \renewcommand{\numero}{D\id.\ano}
                \numberedsections{colorschema}
            }
        }
    }
}


% Change the font of the bibliography
\renewcommand*{\bibfont}{\sffamily\scriptsize}


% Define the header and footer styles for the entire document
\pagestyle{fancy}
\fancyhf{} % Clear existing header and footer
\renewcommand{\headrulewidth}{0pt} % No line in header area
\fancyhead[L]{{
        \fontfamily{phv}
        \selectfont
        {\cabecalho[\docname]{colorschema}}
}} % Use \cabecalho in the left header % Use \cabecalho in the left header
\fancyfoot[C]{\thepage} % center-align the page number in the footer
\setlength{\headheight}{35pt}


%% METADATA ON PDFs
\hypersetup{pdfinfo={
    Title={\titulo},
    Author={\equipaRedatora, \equipaRevisora, \equipaConsultora},
    Subject={\tipodocumento},
    Keywords={\pchave}
}}

\begin{document}
    
    \newgeometry{left=3cm}
\begin{titlepage}
	\pagestyle{plain} % Suppress headers and footers on this page
	\pagecolor{gray1}\afterpage{\nopagecolor}
	
	\begin{tikzpicture}[remember picture,overlay] 
		\node[rounded corners, fill=gray2, anchor=north, minimum width=\paperwidth-1.25cm, minimum height=1.25cm](firstrect) at (current page.north){};%
	\end{tikzpicture}
	
	\begin{tikzpicture}[remember picture, overlay]
		\node[rounded corners, fill=gray2, anchor=north, minimum width=\paperwidth-2cm, 
		minimum height={\paperheight-6cm}](secondrect) 
		at ($(current page.north)-(0,6cm)$) {};
	\end{tikzpicture}
	
	\begin{tikzpicture}[remember picture, overlay]
		\node[anchor=center, inner sep=0pt](logos) at ([yshift=-3.7cm]current page.north) 
		{\includegraphics[width=1\textwidth]{template/logos.png}};
	\end{tikzpicture}
	\vfill
	\vfill

    \begin{flushleft}
		\textcolor{colorschema}{\fontsize{60}{60}\selectfont++}
	\end{flushleft}
	
	\begin{flushleft}
		\fontsize{35}{50}\selectfont\textbf{\docname{} USP-ZP}\\%
		\fontsize{20}{25}\selectfont\textbf{\hspace{1cm}\numero}
	\end{flushleft}
	\vfill
	\vfill
	\vfill
	
	\begin{flushright}
		\textcolor{colorschema}{
			\Huge{\textbf{\titulo}}\\
			\Large{\textbf{Versão \versao}}\\
		}
	\end{flushright}
	

	\begin{flushleft}
		\ifdefstring{\tipodocumento}{recomendacao}{
			%%%%%%%%%%%%%%%%%%%%%%%%%%%%%%%%%%%%%%%%%%%%%%%%%%%%%%%%%%%%%%%%%%%%%%%%%%%%%%%%%%%%%%%%%%%%%%%
%  USP-DOC
%  Template para documentos técnicos das USP
%
%  Copyright (c) 2023 Estevão Soares dos Santos
%%%%%%%%%%%%%%%%%%%%%%%%%%%%%%%%%%%%%%%%%%%%%%%%%%%%%%%%%%%%%%%%%%%%%%%%%%%%%%%%%%%%%%%%%%%%%%%

\fontsize{10}{30}\selectfont{PUBLICAÇÃO: \dataElaboracao}\\
\bigskip
\ifx\dataRevisao\empty
\else
	\fontsize{10}{12}\selectfont{ATUALIZAÇÃO: \dataRevisao}\\
	\bigskip
\fi
\fontsize{10}{12}\selectfont{DESTINATÁRIOS: \destinatarios}\\
\bigskip
\fontsize{10}{12}\selectfont{PALAVRAS-CHAVE: \pchave}\\
		}{
			\ifdefstring{\tipodocumento}{procedimento}{
				%%%%%%%%%%%%%%%%%%%%%%%%%%%%%%%%%%%%%%%%%%%%%%%%%%%%%%%%%%%%%%%%%%%%%%%%%%%%%%%%%%%%%%%%%%%%%%%
%  USP-DOC
%  Template para documentos técnicos das USP
%
%  Copyright (c) 2023 Estevão Soares dos Santos
%%%%%%%%%%%%%%%%%%%%%%%%%%%%%%%%%%%%%%%%%%%%%%%%%%%%%%%%%%%%%%%%%%%%%%%%%%%%%%%%%%%%%%%%%%%%%%%

\fontsize{10}{30}\selectfont{ELABORAÇÃO: \dataElaboracao}\\
\bigskip
\ifx\dataRevisao\empty
\else
	\fontsize{10}{12}\selectfont{REVISÃO: \dataRevisao}\\
	\bigskip
\fi
\ifx\dataAprovacao\empty
\else
	\fontsize{10}{12}\selectfont{APROVAÇÃO: \dataAprovacao}\\
	\bigskip
\fi
\fontsize{10}{12}\selectfont{DESTINATÁRIOS: \destinatarios}\\
\bigskip
\fontsize{10}{12}\selectfont{PALAVRAS-CHAVE: \pchave}\\
			}{
				\ifdefstring{\tipodocumento}{instrucaotrabalho}{
					%%%%%%%%%%%%%%%%%%%%%%%%%%%%%%%%%%%%%%%%%%%%%%%%%%%%%%%%%%%%%%%%%%%%%%%%%%%%%%%%%%%%%%%%%%%%%%%
%  USP-DOC
%  Template para documentos técnicos das USP
%
%  Copyright (c) 2023 Estevão Soares dos Santos
%%%%%%%%%%%%%%%%%%%%%%%%%%%%%%%%%%%%%%%%%%%%%%%%%%%%%%%%%%%%%%%%%%%%%%%%%%%%%%%%%%%%%%%%%%%%%%%

\fontsize{10}{30}\selectfont{PUBLICAÇÃO: \dataElaboracao}\\
\bigskip
\ifx\dataRevisao\empty
\else
	\fontsize{10}{12}\selectfont{ATUALIZAÇÃO: \dataRevisao}\\
	\bigskip
\fi
\fontsize{10}{12}\selectfont{DESTINATÁRIOS: \destinatarios}\\
\bigskip
\fontsize{10}{12}\selectfont{PALAVRAS-CHAVE: \pchave}\\
				}{
					\fontsize{10}{30}\selectfont{DATA: \dataElaboracao}\\
\bigskip
\fontsize{10}{12}\selectfont{DESTINATÁRIOS: \destinatarios}\\
\bigskip
\fontsize{10}{12}\selectfont{PALAVRAS-CHAVE: \pchave}\\
				}
			}
		}
	\end{flushleft}

	

	
\end{titlepage}
\thispagestyle{empty}
\restoregeometry

    % Turn off PDF bookmarking, reset page counter
    \hypersetup{pageanchor=false}
    \setcounter{page}{1}
    
    %%%%%%%%%%%%%%%%%%%%%%%%%%%%%%%%%%%%%%%%%%%%%%%%%%%%%%%%%%%%%%%%%%%%%%%%%%%%%%%%%%%%%%%%%%%%%%%
%  USP-DOC
%  Template para documentos técnicos das USP
%
%  Copyright (c) 2023 Estevão Soares dos Santos
%%%%%%%%%%%%%%%%%%%%%%%%%%%%%%%%%%%%%%%%%%%%%%%%%%%%%%%%%%%%%%%%%%%%%%%%%%%%%%%%%%%%%%%%%%%%%%%

\newgeometry{left=3cm}

\begin{titlepage}
	\pagestyle{plain} % Suppress headers and footers on this page
	\pagecolor{gray2}
	
	\cabecalho[\docname]{colorschema}
	\vfill
	%! suppress = NonMatchingIf
	\ifdefstring{\tipodocumento}{recomendacao}{
		%%%%%%%%%%%%%%%%%%%%%%%%%%%%%%%%%%%%%%%%%%%%%%%%%%%%%%%%%%%%%%%%%%%%%%%%%%%%%%%%%%%%%%%%%%%%%%%
%  USP-DOC
%  Template para documentos técnicos das USP
%
%  Copyright (c) 2023 Estevão Soares dos Santos
%%%%%%%%%%%%%%%%%%%%%%%%%%%%%%%%%%%%%%%%%%%%%%%%%%%%%%%%%%%%%%%%%%%%%%%%%%%%%%%%%%%%%%%%%%%%%%%

\begin{adjustwidth}{0pt}{0.2\textwidth}
	\begin{flushleft}
		\proposta\\
		\bigskip
		\textbf{Redação:}\\
		\equipaRedatora\\
		\bigskip
		\textbf{Revisão:}\\
		\equipaRevisora
	\end{flushleft}
\end{adjustwidth}
	}{
		\ifdefstring{\tipodocumento}{procedimento}{
			%%%%%%%%%%%%%%%%%%%%%%%%%%%%%%%%%%%%%%%%%%%%%%%%%%%%%%%%%%%%%%%%%%%%%%%%%%%%%%%%%%%%%%%%%%%%%%%
%  USP-DOC
%  Template para documentos técnicos das USP
%
%  Copyright (c) 2023 Estevão Soares dos Santos
%%%%%%%%%%%%%%%%%%%%%%%%%%%%%%%%%%%%%%%%%%%%%%%%%%%%%%%%%%%%%%%%%%%%%%%%%%%%%%%%%%%%%%%%%%%%%%%

\vfill
\begin{flushleft}
	\textbf{\textcolor{colorschema}{Ficha Técnica}} \\
	\vspace{12pt}
	
	{
		\renewcommand{\arraystretch}{1.4}
		\begin{tabular}{l!{\color{colorschema}\vrule width 0.5mm}p{0.65\linewidth}}
		    \textbf{Título:}             & \titulo\\
		    \textbf{Número:}             & P\numero\\
		    \textbf{Versão:}             & {\versao} \\
		    \textbf{Local:}              & \local \\
		    \textbf{Data de Elaboração:} & \dataElaboracao \\
		    \textbf{Data de Revisão:}    & \dataRevisao \\
		    \textbf{Data de Aprovação:}  & \dataAprovacao \\
		    \textbf{Equipa Redatora:}    & \equipaRedatora \\
		    \textbf{Equipa Revisora:}    & \equipaRevisora \\
		    \textbf{Equipa Consultora:}  & \equipaConsultora \\
		\end{tabular}
	}
\end{flushleft}
		}{
			\ifdefstring{\tipodocumento}{instrucaotrabalho}{
				\begin{adjustwidth}{0pt}{0.2\textwidth}
	\begin{flushleft}
		\proposta\\
		\bigskip
		\textbf{Redação:}\\
		\equipaRedatora\\
		\bigskip
		\textbf{Revisão:}\\
		\equipaRevisora
	\end{flushleft}
\end{adjustwidth}
			}{
				\begin{adjustwidth}{0pt}{0.2\textwidth}
	\begin{flushleft}
		\proposta\\
		\bigskip
		\textbf{Redação:}\\
		\equipaRedatora\\
		\bigskip
		\textbf{Revisão:}\\
		\equipaRevisora
	\end{flushleft}
\end{adjustwidth}
		}
	}
}
	

	
	
	\begin{tikzpicture}[remember picture, overlay]
		\node[anchor=south, yshift=1cm] at (current page.south) {
			\begin{minipage}{\textwidth}
				\centering
				\small{\contacto}
				\bigskip
				% NÂO REMOVER ESTA COPYRIGHT NOTICE
				\footnotesize{Este documento foi gerado pelo sistema USP-DOC\\ USP-DOC Copyright (c) 2023, Estêvão Soares dos Santos }
				% NÂO REMOVER ESTA COPYRIGHT NOTICE
			\end{minipage}
		};
	\end{tikzpicture}
	\restoregeometry	
\end{titlepage}
\newpage

    
    
    % Turn PDF bookmarking back on and continue with your document
    \hypersetup{pageanchor=true}
    \setcounter{page}{3}
    
    %reset geometry
    \newgeometry{top=4cm, bottom=2cm, left=2cm, right=2cm}
    
    %% LOAD BIBLIOGRAPHY EVEN IF THERE's NO CITATIONS
    \nocite{*}

    
    %% CONDITIONAL LOADING OF TYPES OF DOCS
    
    %! suppress = NonMatchingIf
    \ifdefstring{\tipodocumento}{recomendacao}{
        % recomendação
        %%%%%%%%%%%%%%%%%%%%%%%%%%%%%%%%%%%%%%%%%%%%%%%%%%%%%%%%%%%%%%%%%%%%%%%%%%%%%%%%%%%%%%%%%%%%%%%
%  USP-DOC
%  Template para documentos técnicos das USP
%
%  Copyright (c) 2023 Estevão Soares dos Santos
%%%%%%%%%%%%%%%%%%%%%%%%%%%%%%%%%%%%%%%%%%%%%%%%%%%%%%%%%%%%%%%%%%%%%%%%%%%%%%%%%%%%%%%%%%%%%%%

\section{Enquadramento}\label{sec:enquadramento2}
\setlength{\parskip}{10pt}
\markdownInput{conteudo/recomendacao/enquadramento.md}

\section{Recomendação}\label{sec:recomendacao}
\markdownInput{conteudo/recomendacao/recomendacao.md}

\nocite{*}

\printbibliography[title={Documentos de Referência}]
\pagebreak

    }{
        \ifdefstring{\tipodocumento}{procedimento}{
            % procedimento
            %%%%%%%%%%%%%%%%%%%%%%%%%%%%%%%%%%%%%%%%%%%%%%%%%%%%%%%%%%%%%%%%%%%%%%%%%%%%%%%%%%%%%%%%%%%%%%%
%  USP-DOC
%  Template para documentos técnicos das USP
%
%  Copyright (c) 2023 Estevão Soares dos Santos
%%%%%%%%%%%%%%%%%%%%%%%%%%%%%%%%%%%%%%%%%%%%%%%%%%%%%%%%%%%%%%%%%%%%%%%%%%%%%%%%%%%%%%%%%%%%%%%

\tableofcontents
\newpage


\section{Enquadramento}\label{sec:enquadramento}
\markdownInput{conteudo/procedimento/enquadramento.md}

\section{Objetivo}\label{sec:objetivo}
\markdownInput{conteudo/procedimento/objetivo.md}

\section{Âmbito}\label{sec:ambito}
\markdownInput{conteudo/procedimento/ambito.md}

\section{Destinatários}\label{sec:destinatarios}
\markdownInput{conteudo/procedimento/destinatarios.md}

\section{Definição}\label{sec:definicao}

    \subsection{Definição Funcional}\label{subsec:definicao-funcional}
    \markdownInput{conteudo/procedimento/def-funcional.md}
    
    \subsection{Limites de Entrada no Processo}\label{subsec:limites-de-entrada-no-processo}
    \markdownInput{conteudo/procedimento/limites-entrada.md}

    \subsection{Limites de Saída do Processo}\label{subsec:limites-de-saida-do-processo}
    \markdownInput{conteudo/procedimento/limites-saidal.md}

    \subsection{Limites Marginais}\label{subsec:limites-marginais}
    \markdownInput{conteudo/procedimento/limites-marginais.md}

\section{Responsabilidades}\label{sec:responsabilidades}
\markdownInput{conteudo/procedimento/responsabilidades.md}

\section{Descrição do procedimento}\label{sec:descricao-do-procedimento}
\markdownInput{conteudo/procedimento/descricao.md}

\section{Situações de Excepção}\label{sec:situacoes-de-excepcao}
\markdownInput{conteudo/procedimento/excepcoes.md}

\section{Monitorização e Avaliação}\label{sec:monitorizacao-e-avaliacao}
\markdownInput{conteudo/procedimento/monitorizacao.md}

\printbibliography[title={Documentos de Referência}]

\newpage

% changelog
\section{Registo de Alterações}\label{sec:registo-de-alteracoes}
\stoptocentries% Stop adding content to the ToC

% Define section and subsection titles without numbering
\titleformat{\section}[block]{\normalfont\Large\bfseries}{}{0em}{}
\titleformat{\subsection}[block]{\normalfont\large\bfseries}{}{0em}{}

\markdownInput{changelog.md}

% Restore the original section and subsection titles
\titleformat{\section}[block]{\normalfont\Large\bfseries}{\thesection}{1em}{}
\titleformat{\subsection}[block]{\normalfont\large\bfseries}{\thesubsection}{1em}{}

\starttocentries% Resume adding content to the ToC


        }{
            \ifdefstring{\tipodocumento}{instrucaotrabalho}{
                % instrução de trabalho
                \section{Enquadramento}
\markdownInput{conteudo/instrucaotrabalho/enquadramento.md}

\section{Objetivo}
\markdownInput{conteudo/instrucaotrabalho/objetivo.md}

\printbibliography[title={Documentos de Referência}]

\section{Responsabilidades}
\markdownInput{conteudo/instrucaotrabalho/responsabilidades.md}

\section{Procedimentos}
\markdownInput{conteudo/instrucaotrabalho/procedimentos.md}

\section{Formulários e Modelos}
\markdownInput{conteudo/instrucaotrabalho/formsmodelos.md}

\section{Monitorização}
\markdownInput{conteudo/instrucaotrabalho/monitorizacao.md}
            }{
                \ifdefstring{\tipodocumento}{checklist}{
                    % instrução de trabalho
                    %%%%%%%%%%%%%%%%%%%%%%%%%%%%%%%%%%%%%%%%%%%%%%%%%%%%%%%%%%%%%%%%%%%%%%%%%%%%%%%%%%%%%%%%%%%%%%%
%  USP-DOC
%  Template para documentos técnicos das USP
%
%  Copyright (c) 2023 Estevão Soares dos Santos
%%%%%%%%%%%%%%%%%%%%%%%%%%%%%%%%%%%%%%%%%%%%%%%%%%%%%%%%%%%%%%%%%%%%%%%%%%%%%%%%%%%%%%%%%%%%%%%

\tableofcontents
\newpage

\section{Lista de Verificação}\label{sec:lista-de-verificacao}
\begin{small}
\begin{luacode}
dofile("template/utils/process_checklist.lua")
process_checklist("conteudo/checklist/checklist.md")
\end{luacode}
\end{small}
\newpage

\section{Guião da Lista de Verificação}\label{sec:guiao-da-lista-de-verificacao}
\markdownInput[startNumber=true]{conteudo/checklist/guiao.md}
\newpage

\printbibliography[title={Referências}]
                }{
                    % default (Documento)
                    \markdownInput[shiftHeadings=0]{conteudo/documento/documento.md}
                }
            }
        }
    }
    
    \pagebreak
    \begin{luacode}
        dofile('template/utils/directory.lua')

        if (Directory.exists('conteudo/anexos') == true and not Directory.is_empty('conteudo/anexos')) then
            tex.print('\\section{Anexos}')
            Directory.tex.include_all_tex_files('./conteudo/anexos')
        end

    \end{luacode}

\end{document}
