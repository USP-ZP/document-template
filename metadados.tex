%%%%%%%%%%%%%%%%%%%%%%%%%%%%%%%%%%%%%%%%%%%%%%%%%%%%%%%%%%%%%%%%%%%%%%%%%%%%%%%%%%%%%%%%%%%%%%%
%  USP-DOC
%  Template para documentos técnicos das USP
%
%  Copyright (c) 2023 Estevão Soares dos Santos
%%%%%%%%%%%%%%%%%%%%%%%%%%%%%%%%%%%%%%%%%%%%%%%%%%%%%%%%%%%%%%%%%%%%%%%%%%%%%%%%%%%%%%%%%%%%%%%


%  NOTA: Nem todos os tipos de documentos usam todos os tipos de metadados

% escolher o tipo de documento apropriado, removendo o símbolo de comentário (%) do inicio da linha
% e comentando as restantes
%\newcommand{\tipodocumento}{documento}          % documento genérico
%\newcommand{\tipodocumento}{recomendacao}       % recomendação
%\newcommand{\tipodocumento}{procedimento}       % procdimento
%\newcommand{\tipodocumento}{instrucaotrabalho}  % instrução de trabalho
\newcommand{\tipodocumento}{checklist}           % lista de verificação


% Variáveis do documento (alterar conforme necessário)
\def\titulo{Template para documento da USP com um título que pode ser bastante longo}

% O número do procedimento deve ter o seguinte formato <letra>ID.ANOvMAJOR.MINOR.PATCH
%  - letra: Letra correspondente ao tipo de documento.
%           (D: documentos; R: recomendação; P: procedimento; I: Instrução de trabalho; C: Lista de verificação)
%  - ID:    Número composto por 3 dígitos (ex: 001). O ID é incremental e faz reset em cada ano
%  - ANO:   Ano civil em que o procedimento foi criado pela primeira vez
%  - MAJOR: Versão major do documento. Corresponde a uma grande revisão do documento.
%           Geralmente implica nova revisão cientifica extensa
%  - MINOR: Versão minor do documento. Corresponde a uma pequena revisão do documento,
%           com alterações "cirurgicas" da recomendação
%  - PATCH: Versão patch do documento. Corresponde a pequenas alterações como correcção de
%           gralhas ou erros ortográficos
\def\id{001}
\def\ano{2023}
\def\major{1}
\def\minor{0}
\def\patch{0}

% palavras chave
\def\pchave{Rastreio, Enterobacterales produtores de carbapenemases, \emph{Staphylococcus aureus} resistente a 
	meticilina, admissão hospitalar, internamento} 

% destinatarios
\def\destinatarios{Profissionais de Saúde do ACES Oeste Norte}

\def\local{USP do ACES Oeste Norte}

\def\dataElaboracao{2023-09-01}
\def\dataRevisao{2023-12-31}
\def\dataAprovacao{2023-12-31}

\newcommand{\proposta}{
O presente documento foi elaborado por proposta da conjunta da 
Unidade de Saúde Pública do ACES Oeste Norte,
o Centro Hospitalar do Oeste e do Departamento de Saúde Pública
da Administração Regional de Saúde de Lisboa e Vale do Tejo.
}

\def\equipaRedatora{
	João Silva, Maria Fernandes, Bruno Almeida, Ana Luísa Pereira
}
\def\equipaRevisora{
	Tiago Carvalho, Sofia Martins, André Lopes Rocha
}
\def\equipaConsultora{
	Rita Oliveira Santos e Pedro Gonçalves
}

\def\contacto{
	Unidade de Saúde Pública Zé-Povinho \\
	Rua Etelvino Santos, s/n.º;
	2500-297 Caldas da Rainha\\
	Telefone n.º 262 248 840\\
	usp.oestenorte@arslvt.min-saude.pt\\
}
